\documentclass{article}
\usepackage[utf8]{inputenc}
\usepackage[T1]{fontenc}
\usepackage[a4paper, total={480pt, 720pt}]{geometry}
\usepackage{array}
\usepackage[polish]{babel}
\renewcommand{\arraystretch}{1.25}

\usepackage[lastpage,user]{zref}
\usepackage{booktabs}
\usepackage{calc}
\usepackage{enumitem}
\usepackage{fancyhdr}
\usepackage{fancyvrb}
\usepackage{float}
\usepackage{graphicx}
\usepackage{hyperref}
\usepackage{longtable}
\usepackage{scalerel}
\usepackage{tabularx}
\usepackage{xparse}
\hypersetup{colorlinks=true, urlcolor=blue}

\let\oldverbatim\verbatim
\let\oldendverbatim\endverbatim
\def\verbatim{\Verbatim[samepage=true]}
\def\endverbatim{\endVerbatim}

\let\OldIncludegraphics\includegraphics
\renewcommand{\includegraphics}[2][]{\OldIncludegraphics[scale=0.55, #1]{#2}}

\providecommand{\tightlist}{%
    \setlength{\itemsep}{0pt}\setlength{\parskip}{0pt}}

\setcounter{secnumdepth}{0}

\def \problemName           {<<<name>>>}
\def \memoryLimit           {<<<mem>>>}
\def \additionalFooterInfo  {}
\def \contestName           {}
\def \roundDate             {}

\pagestyle{fancy}
\fancyhf{}
\renewcommand{\headrulewidth}{0pt}
\renewcommand{\footrulewidth}{0.5pt}
\lfoot{\additionalFooterInfo}
\cfoot{\thepage/\zpageref{LastPage}}
\rfoot{\problemName}

\begin{document}
\section*{\huge\problemName\hfill\small{Dostępna pamięć: \memoryLimit}\newline
\rule{480pt}{0.5pt}
%\newline\small\contestName\hfill\normalfont\textit\roundDate}
}

\setlength{\parindent}{0pt}
\setlength{\parskip}{0.4\baselineskip}

\begin{sloppypar}

Problem statement.

\section{Wejście}

Input description.

\section{Wyjście}

Output description.

\section{Przykłady}

\ExplSyntaxOn
\NewDocumentCommand{\readcell}{m}
{
    \sip_statement_readcell:n { #1 } % #1 is the file to read
    \vspace{-\baselineskip} % remove empty line that results from the trailing \newline
}
% some variables
\tl_new:N \l__sip_statement_cell_content_tl
\ior_new:N \g_sip_statement_import_stream
% proper implementation
\cs_new_protected:Npn \sip_statement_readcell:n #1
{
    % clear the variable containing the cel contents
    \tl_clear:N \l__sip_statement_cell_content_tl
    % start reading the file
    \ior_open:Nn \g_sip_statement_import_stream { #1 }
    % at each line ...
    \ior_map_inline:Nn \g_sip_statement_import_stream
    {
        % add the line and a trailing \newline
        \tl_put_right:Nn \l__sip_statement_cell_content_tl { ##1 \newline }
    }
    \tl_use:N \l__sip_statement_cell_content_tl % deliver the result
}
\ExplSyntaxOff
\NewDocumentCommand{\testFiles}{mm}{
    \readcell{#1} & \readcell{#2} \\ \hline
}
\NewDocumentCommand{\standardTestFiles}{m}{
    \testFiles{in/#1.in}{out/#1.out}
}

\begin{tabularx}{\textwidth}{*{2}{|>{\ttfamily}X}|}
    \hline
        \multicolumn{1}{|c|}{\textbf{Wejście}} & \multicolumn{1}{c|}{\textbf{Wyjście}}
    \\ \hline
    %%% The lines below are equivalent, choose any:
    % \readcell{in/0.in} & \readcell{out/0.out} \\ \hline
    % \testFiles{in/0.in}{out/0.out}
    % \standardTestFiles{0}
    \standardTestFiles{0}
\end{tabularx}

\section{Ocenianie}
\newcounter{subtask}\NewDocumentCommand{\subtask}{mm}{
    \stepcounter{subtask}
    \thesubtask & #1 & #2 \\ \hline
}
\begin{center}
    \begin{tabularx}{0.8\textwidth}{|c|X|c|}
        \hline
            \multicolumn{1}{|c|}{\textbf{Podzadanie}} & \multicolumn{1}{c|}{\textbf{Ograniczenia}} & \multicolumn{1}{c|}{\textbf{Punkty}}
        \\ \hline
        \subtask{$n \leq 10$}{40}
        \subtask{$n \leq 1000$}{30}
        \subtask{brak dodatkowych ograniczeń}{30}
    \end{tabularx}
\end{center}

\end{sloppypar}
\end{document}
