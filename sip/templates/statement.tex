\documentclass{article}
\usepackage[utf8]{inputenc}
\usepackage[T1]{fontenc}
\usepackage[a4paper, total={480pt, 720pt}]{geometry}
\usepackage{array}
\usepackage[polish]{babel}
\renewcommand{\arraystretch}{1.25}

\usepackage{tabularx}
\usepackage{fancyhdr}
\usepackage[lastpage,user]{zref}
\usepackage{enumitem}
\usepackage{graphicx}
\usepackage{float}
\usepackage{calc}
\usepackage{scalerel}
\usepackage{fancyvrb}
\usepackage{longtable}
\usepackage{booktabs}
\usepackage{hyperref}
\hypersetup{colorlinks=true, urlcolor=blue}

\let\oldverbatim\verbatim
\let\oldendverbatim\endverbatim
\def\verbatim{\Verbatim[samepage=true]}
\def\endverbatim{\endVerbatim}

\let\OldIncludegraphics\includegraphics
\renewcommand{\includegraphics}[2][]{\OldIncludegraphics[scale=0.55, #1]{#2}}

\providecommand{\tightlist}{%
    \setlength{\itemsep}{0pt}\setlength{\parskip}{0pt}}

\setcounter{secnumdepth}{0}

\def \problemName           {<<<name>>>}
\def \memoryLimit           {<<<mem>>>}
\def \additionalFooterInfo  {}
\def \contestName           {}
\def \roundDate             {}

\pagestyle{fancy}
\fancyhf{}
\renewcommand{\headrulewidth}{0pt}
\renewcommand{\footrulewidth}{0.5pt}
\lfoot{\additionalFooterInfo}
\cfoot{\thepage/\zpageref{LastPage}}
\rfoot{\problemName}

\begin{document}
\section*{\huge\problemName\hfill\small{Dostępna pamięć: \memoryLimit}\newline
\rule{480pt}{0.5pt}
%\newline\small\contestName\hfill\normalfont\textit\roundDate}
}

\setlength{\parindent}{0pt}
\setlength{\parskip}{0.4\baselineskip}

\begin{sloppypar}

Problem statement.

\section{Wejście}

Input description.

\section{Wyjście}

Output description.

\section{Przykłady}
\begin{tabularx}{\textwidth}{*{2}{|>{\ttfamily}X}|}
	\hline
		\multicolumn{1}{|c|}{\textbf{Wejście}}
	&
		\multicolumn{1}{c|}{\textbf{Wyjście}}
	\\ \hline
		a\newline
		b
	&
		out a b
	\\ \hline
		c\newline
		d
	&
		out c d
	\\ \hline
\end{tabularx}

\section{Ocenianie}
\begin{center}
    \begin{tabularx}{0.8\textwidth}{|c|X|c|}
        \hline
            \multicolumn{1}{|c|}{\textbf{Podzadanie}}
        &
            \multicolumn{1}{c|}{\textbf{Ograniczenia}}
        &
            \multicolumn{1}{c|}{\textbf{Punkty}}
        \\ \hline
            1
        &
            $n \leq 10$
        &
            40
        \\ \hline
            2
        &
            $n \leq 1000$
        &
            30
        \\ \hline
            3
        &
            brak dodatkowych ograniczeń
        &
            30
        \\ \hline
    \end{tabularx}
\end{center}

\end{sloppypar}

\end{document}
